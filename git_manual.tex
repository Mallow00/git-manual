\documentclass[12pt, a4paper]{article}

% PACCHETTI
\usepackage[italian]{babel}
\usepackage[utf8]{inputenc}
\usepackage{hyperref}
\usepackage{fancyhdr}
\usepackage[T1]{fontenc}
\usepackage[top=1.5cm, bottom=3.5cm, left=1.5cm, right=1.5cm, ]{geometry}
\usepackage{lastpage}
\usepackage{tikz}
\usepackage{xcolor}

% INFORMAZIONI DOCUMENTO
\title{Git Manual}
\author{LuBu}
\date{}

% STILE PAGINE
\pagestyle{fancy}
\setcounter{page}{0}

% HEADER
\setlength{\headheight}{50pt}
\lhead{\texttt{GIT}}
\rhead{\includegraphics[width=1cm]{./images/icon.png}}
\renewcommand{\headrulewidth}{.5pt} % MARGINI HEADER

% FOOTER
\cfoot{}
\rfoot{\texttt{Pagina \thepage~di~\pageref{LastPage}}}
\renewcommand{\footrulewidth}{.5pt} % MARGINI FOOTER

% DOCUMENTO
\begin{document}

% PRIMA PAGINA
\maketitle
\tikz[remember picture,overlay] \node[opacity=1, inner sep=0pt] at (current page.center){\includegraphics[width=250pt ,height=250pt]{./Images/icon.png}};
\thispagestyle{empty}
\clearpage
\noindent
\vspace{2.5 cm}
\begin{center}
    \textbf{Sommario}\\
    Questo è un manuale introduttivo a \texttt{GIT} scritto durante le lezioni di \texttt{Metodi e Tecnologie per lo Sviluppo Software}, a.a. 2021/22
\end{center}
\clearpage
\tableofcontents
\clearpage

\section{Introduzione}
Git è un software di controllo versione distribuito utilizzabile da interfaccia a riga di comando, creato da Linus Torvalds nel 2005.\\
La maggior parte delle operazioni viene fatta in locale.\\
Distribuito per “clone” del repository, permette backups multipli e la possibilità di adottare diversi Work flow.\\
Ogni commit è identificato da un ID (checksum SHA-1 di 40-caratteri basato sul contenuto di file o della struttura della directory) che ne garantisce l’integrità.
Non è possibile cambiare un commit senza modificare l’ID del commit stesso e di i commit successivi

\subsection{Staging Area}
È stata aggiunta un’area di staging dove vengono validati i file modificati che potranno essere versionati con un commit.\\\\
In GIT i file della copia locale possono essere:
\begin{itemize}
    \item Nella \textbf{Working directory} Checked out, modificati ma non ancora validati (Modified);
    \item Nella \textbf{Staging Area}, validati ma non ancora committati. Il commit salva uno snapshot di tutti i file presenti nella staging area (Staged);
    \item Nel \textbf{Repository locale} (Committed)
\end{itemize}

\subsection{Stato di un file in GIT}
Un file in GIT può essere in uno dei seguenti 3 stati:
\begin{itemize}
    \item \textbf{Modifed:} modificato nella working directory;
    \item \textbf{Staged:} salva una snapshot nella staging area;
    \item \textbf{Committed:} preso dalla staging area e salvato nel repository locale.
\end{itemize}

\subsection{Configurazione}
E’ richiesta una configurazione iniziale dove vengono impostati l’username e l’email da usare per ogni commit:
\begin{center}
    \begin{itemize}
        \item \texttt{git config --global user.name "username"}
        \item \texttt{git config --global user.email user\char`_email@example.com}
    \end{itemize}
\end{center}
E’ possibile invocare il seguente comando per avere la lista di tutte le configurazioni:
\begin{center}
    \texttt{git config --list}
\end{center}

Le configurazioni possono essere fatte a vari livelli:
\begin{center}
    \begin{itemize}
        \item \textbf{system:} per l’intero sistema per tutti gli utenti
        \item \textbf{global:} per il singolo utente
        \item \textbf{local (di default):} per singolo repository
    \end{itemize}
\end{center}

\subsection{Creazione del repository}
Due scenari:\\\\
\textbf{Creazione del repository locale nella cartella corrente:}
\begin{center}
    \texttt{git init}
\end{center}
Crea una cartella .git nella cartella corrente. Da questo momento è possibile iniziare il versionamento dei file localmente.\\\\
\textbf{Clonazione di un repository remoto nella cartella corrente:}
\begin{center}
    \texttt{git clone} \textbf{urlLocalDirectoryName}
\end{center}
Clona il contenuto del repository remoto nella cartella corrente e crea una cartella .git che rappresenta il repository locale

\clearpage

\section{Comandi}
\subsection{Comandi Base}
\begin{itemize}
    \item \texttt{git init} inizializza la cartella
    \item \texttt{git add} aggiunge i file nella staging area e crea una snapshop. In particolare:
    \begin{itemize}
        \item[] \texttt{git add 'nomeFile'} aggiunge il file 'nomeFile' alla staging area (è possibile indicare più file da aggiungere)
        \item[] \texttt{git add *.estensione} aggiunge tutti i file con estensione .estensione alla repo git
        \item[] \texttt{git add .} aggiunge tutti i file che hanno subito modifiche e i file precedentemente non tracciati alla staging area
    \end{itemize}
    \item \texttt{git commit -m 'Titolo'} serve ad eseguire il commit e a dare un titolo (posso aggiungere una descrizione con git commit -m “Titolo” -m “descrizione”)
    \item \texttt{git restore --staged 'nomefile'} rimuove il file 'nomefile' dalla repo git
    \item \texttt{git commit --amend} modifica titolo/descrizione dell'ultimo commit
    \item \texttt{git reset} serve ad eliminare i commit. In particolare:
    \begin{itemize}
        \item[] \texttt{}{git reset --soft} riporta lo stato alla staging area precedente al commit.
        \item[] \texttt{git reset --mixed} riporta lo stato preccedente alla staging area (default).
        \item[] \texttt{git reset --hard} cancella sia il commit che eventuali file aggiunti non presenti ad un commit precedente dalla working directory.
    \end{itemize}
    \textbf{N.B.} vicino al comando \texttt{reset} oltre alla modalità si deve indicare \texttt{HEAD} + uno tra \texttt{'id-commit', \char`^\ , $\sim$}:
    \begin{itemize}
        \item[] \texttt{'id commit'} riporta lo stato al commit di cui è stato indicato l’id.
        \item[] \texttt{\char`^} torna indietro di un commit per ciascun \char`^\ presente.
        \item[] \texttt{$\sim$\#commit} torna indietro di un numero di commit pari a quello indicato dal numero insetiro.\\\\
    \end{itemize}
    \textbf{ESEMPIO:}\\
    \texttt{git reset --soft HEAD\char`^\char`^} → torna indietro di 2 commit riportando i file non prensenti al commit in questione alla staging area.\\
    \texttt{git reset --hard HEAD$\sim$3} → torna indietro di 3 commit eliminando le modifiche non prensenti nel commit al quale si sta tornando.\\
    
\end{itemize}

\subsection{Comandi di stato e ripristino delle modifiche}
\begin{itemize}
    \item \texttt{git status} mostra lo stato dei file all’interno del progetto\\
    \texttt{git status -s} (short version) mostra lo stato dei file all’interno del progetto con una versione più breve
    \item \texttt{git diff} Per vedere cos’è stato modificato ma non ancora validato nella staging area
    \item \texttt{git diff -cached} Per vedere cos’è stato modificato nella staging area
    \item \texttt{git checkout -- 'nomeFile'} serve a rimuovere le modifiche dal file 'nomeFile'.
    \item \texttt{git restore 'nomeFile'} serve a rimuovere le modifiche dal file 'nomeFile'. (Stesso comportamento del comando precedente).
    \item \texttt{git log} mostra i conti eseguiti fino a questo momento e le loro informazioni (id, titolo, descrizione, autore, data).\\
    Si possono utilizzare opzioni quali \texttt{--oneline} per delle informazioni più compatte, \texttt{--reverse} per vederle in ordine inverso\\
    \texttt{git log -2} Per vedere le ultime 2 modifiche

\end{itemize}

\subsection{Comandi Branch}
\begin{itemize}
    \item \texttt{git branch} mostra i branch esistenti nel progetto. Il branch su cui ci troviamo al momento corrente è indicato da un *
    \item \texttt{git branch 'nomeBranch'} serve per creare un nuovo branch con nome 'nomeBranch'
    \item \texttt{git checkout 'nomeBranch'} sposta il puntatore HEAD verso il branch 'nomeBranch', quindi in parole povere serve a spostarsi tra i vari brach.
    \item \texttt{Git checkout -b 'nomeBranch'} crea un nuovo branch 'nomeBranch' e sposta il puntatore HEAD sullo stesso
    \item \texttt{git merge 'nomeBranch'} serve ad unire il branch 'nomeBranch' nel branch corrente
    \item \texttt{git branch --merged} serve a vedere quali branch sono stati uniti
    \item \texttt{git branch -M 'nomeBranch'} serve per rinominare il branch corrente in 'nomeBranch'
    \item \texttt{git branch -d 'nomeBranch'} serve ad eliminare il branch 'nomeBranch'. \textbf{Funziona solo se il branch è stato unito ad un altro branch}
    \item \texttt{git branch -D 'nomeBranch'} serve ad eliminare il branch 'nomeBranch'
    \item \texttt{git branch --abort} serve ad interrompere il merge pendente tra due branch
\end{itemize}

\subsection{Repository Remoto}
\begin{itemize}
    \item \texttt{git remote add 'nomeServer' 'urlServer'} serve per inserire l’indirizzo di un server remoto a cui effettuare push e pull
    \item \texttt{git remote} fornisce la lista dei nomi dei server remoti registrati in precedenza.
    \item \texttt{git remote -v} fornisce la lista dei nomi e degli url dei server remoti registrati in precedenza
    \item \texttt{git remote show 'nomeServer'} Serve ad ispezionare la configurazione del server 'nomeServer'
    \item \texttt{git remote rename 'nomeServer' 'nuovoNomeServer'} serve a rinominare il server 'nomeServer' in 'nuovoNomeServer'
    \item \texttt{git remote remove 'nomeServer'} serve ad eliminare il server
\end{itemize}

\subsection{Sincronizzazione}
\begin{itemize}
    \item \texttt{git push -u 'nomeServer' 'nomeBranch'} serve per fare l’upload del nostro codice ad un server remoto. L’opzione -u serve a specificare nella configurazione un server remoto di default
    \item \texttt{git fetch} sincronizza le informazioni della repository remota nella repository locale. Dopo è necessario effettuare il comando git merge per unire i commit
    \item \texttt{git pull} sincronizza le informazioni della repository remota nella repository locale unendo automaticamente i commit
    \item \texttt{git clone 'urlServer'} clona il remote server 'urlServer' in locale, mantenendo tutti i commit e le modifiche nel tempo
\end{itemize}
Il comando clone accetta anche alcune opzioni \textbf{ad esempio:}\\
\texttt{git clone 'urlServer' 'nomeCartella'} clona il server in locale assegnando il nome 'nomeCartella' alla repository appena clonata
\clearpage

%%%%%%%%%%%%%%%%%%%%%%%%%%%
\section{GitFlow}
Giflow è un modello di ramificazione Git alternativo che prevede l'uso di rami di feature e più rami primari. Dopo l'installazione è possibile eseguire dei comandi che combinano quelli normali di git in altri più compatti favorendo il tipo di workflow precedentemente descritto.\\
Nel seguito una breve descrizione dei comandi:
\begin{itemize}
    \item[] \texttt{git-flow init} inizializza git-flow all'interno di un repo git esistente per iniziare a usarlo.\\
    Dopo aver lanciato il comando sarà necessario rispondere ad alcune domande riguardanti le convenzioni dei nomi per i branch (si consiglias di mantenere i valori di default).\\
    Una volta terminato di rispondere alle domande ci troveremo già nel branch \texttt{develop} creato a partire da \texttt{master}
    \item[] \texttt{git-flow feature 'featureName'} crea un nuovo feature-branch a partire da \texttt{develop} e sposta HEAD sul branch appena creato.
    \item[] \texttt{git-flow feature finish 'featureName'} effettua il merge del feature-branch nel branch \texttt{develop}, effettua il checkout in \texttt{develop} ed elimina il feature-branch.
    \item[] \texttt{git-flow feature publish 'featureName'} effettua il \texttt{push} del branch al server remoto.
    \item[] \texttt{git-flow feature pull origin 'featureName'} effettua il \texttt{pull} del branch dal server remoto
    \item[] \texttt{git flow release start 'RELEASE'} crea un release-branch a partire da develop
    \item[] \texttt{git flow release publish 'RELEASE'} effettua il \texttt{push} del branch al server remoto.
    \item[] \texttt{git flow release finish 'RELEASE'} effettua il merge del release-branch in \texttt{master} e anche in \texttt{develop} taggando la release con il suo nome ed elimina il release-branch.
\end{itemize}

%%%%%%%%%%%%%%%%%%%%%%%%%%%

\section{Gitignore}
Il comando \texttt{touch .gitignore} crea un file \textbf{.gitignore} che ignora i file e le cartelle contenuti in esso dalla funzione di add e di conseguenza dai commit.\\
È possibile aprire il file .gitignore con un \textbf{editor di testo}.\\
All’interno del file .gitignore è possibile scrivere:
\begin{itemize}
    \item \textbf{commenti a linea singola} iniziando la riga con il carattere \texttt{\#}
    \item \textbf{nomi dei file/cartelle}che si desidera ignorare
    \item \textbf{*.estensione} per rimuovere tutti i file con l’estensione indicata\\
    \texttt{!'nomeFile'} permette di rendere visibile un file la cui estensione è indicata all’interno del file .gitignore di cui però si vuole mantenere la visibilità
\end{itemize}

\vspace{2.5 cm}
\noindent
\vspace{0.8 cm}
\noindent
\textbf{Esempio di file .gitignore}\\
\texttt{\textcolor{teal}{\#directories}}\\
\texttt{ignorami}\\
\texttt{ignorami\_due}\\\\
\texttt{\textcolor{teal}{\#Files}}\\
\texttt{ignorami.jpg}\\
\texttt{ignorami.png}\\
\texttt{ignorami.txt}\\
\texttt{ignorami.mp3}\\\\
\texttt{\textcolor{teal}{\#Extensions}}\\
\texttt{*.gz}\\
\texttt{*.aux}\\
\texttt{*.fdb\_latexmk}\\
\texttt{*.fls}\\
\texttt{*.out}\\
\texttt{*.toc}\\
\texttt{*.log}\\\\
\texttt{\textcolor{teal}{\#macOS}}\\
\texttt{.DS\_Store}\\

\clearpage

\appendix

\section{Utilities}
\subsection{Work Flow}
\begin{itemize}
    \item[] \textbf{Centralized Work Flow:} tutti gli sviluppatori lavorano in un unico branch. Sicuramente si andrà spesso incontro a conflitti.
    \item[] \textbf{Feature Branch Work Flow:} viene creato un branch apposito per ciascuna feature.
\end{itemize}

\end{document}%%%END